\section{Цели и задачи проекта}
	Главный проект декомпозируется представителями всех команд на составляющие, одной из которых и является ML-часть, выделяющаяся в отдельный Git-проект. ML-проект так же должен быть разбит на элементы. В идеале \textit{\textbf{составляющие ML-проекта должны быть сведены к давно известным задачам, решение которых является стереотипным}}: детекция, трекинг, сегментация, классификация, регрессия, фильтрация и т.п.

\subsection{Классификация ML-задачи}
	От типа решаемой ML-задачи зависят дальнейшие шаги. Без этого невозможно даже формализовать требования к разметке, не говоря уже о последующих этапах.

	В первом приближении декомпозиция ML-проекта должна сводить его элементы к использованию хорошо известных и успешно апробированных в мировой практике решений. Любое отклонение от этого переводит проект из разряда чисто инженерных в исследовательские, что влечёт дополнительные риски. Всё это необходимо учитывать ещё на этапе декомпозиции главной задачи.

	Стереотипными CV-задачами являются: сегментация, детекция, трекинг, фильтрация, классификация, регрессия и т.п.

\subsection{Выбор модели подходящего семейства}
	Если ML-задачу удалось свести к стереотипному решению, то оно автоматически диктует: архитектуру модели, пред- и постобработки данных, метрики и открытые датасеты. Например, для сегментации подходят U-Net или DeepLab, а для детекции - YOLO или R-CNN.Так же в большинстве случаев имеется возможность выбирать базовую модель (backbone) для извлечения признаков из изображений. Однако, использование моделей на базе трансформеров обычно не представляется возможным ввиду ограничений бортового оборудования и необходимости работы в реальном времени. Но определять модель настолько детально на этом этапе не требуется, т.к. впереди должны быть ещё исследования соотношений точности/скорости для различных гиперпараметров. Следует лишь иметь это ввиду и учитывать возможности выбранной аппаратной платформы.

	Если для ML-задачи было выбрано нетипичное решение, следует выделить дополнительные ресурсы на проведение исследований. Это в любом случае потребует времени и резко увеличит риски провалить проект или завершить его позже планируемого срока.

\section{Создание датасета}
	Исходя из типа задачи и формата эталонных данных можно в первом приближении определить что за данные необходимо собрать и разметить.

	Большинство проблем почти любого ML-проекта (особенно на ранних стадиях) связано с недостатками датасета:
	\begin{enumerate}
		\item беден даже с учётом аугментации или вообще отсутствует;
		\item нерелевантен (не в полной мере соответствует задаче, условиям наблюдения, типу используемых датчиков и т.п.);
		\item противоречиво, неполно или неточно размечен;
	\end{enumerate}

	Сам процесс сбора новых данных и их разметка так же могут вызывать трудности:
	\begin{enumerate}
		\item дороговизна получения новых данных;
		\item привязка к независящим от человека условиям (например, погода) или редкость фиксируемого явления;
		\item высокие требования к квалификации разметчика или трудоёмкость самой разметки;
		\item юридические ограничения (персональные данные, секретность и т.п.).
	\end{enumerate}

	Для компенсации указанных проблем зачастую прибегают к компьютерному моделированию. Это может быть разработка полноценного симулятора с нуля или модификация какого-то из существующих решений. К тому же \textit{\textbf{полноценный симулятор снимает необходимость размечать синтетические данные и открывает большое количество возможностей по тестированию}}. Но и это направление сопряжено с рядом проблем:
	\begin{enumerate}
		\item необходимость иметь команду соответствующих разработчиков в штате или на аутсорсе;
		\item труднодостижимый фотореализм;
		\item сложная механика моделирования динамических систем, особенно относящихся к передвижению носителя или ОИ;
		\item некоторым задачам требуется большое количество 3D-моделей для наполнения фона или разнообразия ОИ, которые не всегда можно найти в открытых асетах.
	\end{enumerate}
	Если разработать полноценный симулятор нет возможности, то прибегают к созданию относительно релевантного набора данных, синтезированного иным образом. Например, с использованием какой-либо компьютерной игры или генеративной НС. В крайнем случае модифицируют имеющиеся данные в полуручном режиме. Но при всех этих вариантах, во-первых, недоступно полноценное моделирование работы замкнутой системы, а во-вторых, сохраняется необходимость размечать данные в ручном режиме.

\subsection{Оптимизация процессов формирования датасета}
	Из-за всех вышеперечисленных проблем процесс сбора и/или разметки данных почти всегда крайне обременителен для проекта. На этапах от POC до MVP вообще стараются использовать датасеты из открытых источников, даже если они соответствуют задаче с большой натяжкой. Зрелые проекты почти всегда отличаются хорошо продуманным и отлаженным процессом непрерывного пополнения датасета. Переход от вакуума к эффективно выстроенному конвейеру обработки данных должен быть эволюционным, а не революционным. При любом изменении в архитектуре такой системы руководствоваться надо исключительно эффектом, который оно окажет на результат обучения модели. Сначала создаётся самое необходимое для работы, и постепенно добавляются всё менее остро необходимые элементы системы. Узкие места процесса могут быть совершенно разными от проекта к проекту. Узких мест обычно всего два: сбор первичных данных и разметка.
	\subsubsection{Нехватка сырых данных}
	\label{subsubsec:slowdataacquisition}
	Получать релевантные данные может быть крайне затруднительно или дорого. В таких случаях следует начать со сбора и разметки менее релевантных данных, параллельно разрабатывая свой симулятор, если на долгой дистанции он окупит вложения. При этом те немногие данные, что имеются сейчас, должны использоваться в первую очередь в проверочной и тестовой выборках.

	Если нужными данными располагает некая третья сторона, стоит поискать возможность заинтересовать её в предоставлении этих данных. Например, создать онлайн-сервис по обработке этих данных, который закроет какие-то из потребностей этой стороны.	По этой схеме работают психологические тесты в соц. сетях, которые требуют доступ к страничке с личными данными испытуемого. Бесплатный доступ к языковым моделям тоже делается в обмен на право использовать всю накопленную переписку.
	\subsubsection{Низкая скорость разметки}
	\label{subsubsec:slowlabeling}
	Объём поступающих на разметку данных может кратно или даже на порядки превышать возможности команды разметки. В этом случае необходимо выстроить конвейер фильтров, каждый из которых отбраковывает какую-то часть данных. Начинать надо с создания самых поверхностных фильтров, имеющих максимальный объём отбракованных данных при минимальном задействовании людей. Например, вести поиск повторяющихся кадров или удаление продолжительных фрагментов видео, где нет изменений в кадре.

	Далее можно ускорить разметку путём использования фундаментальных CV-моделей для выполнения предварительной автоматизированной разметки (предразметки). Для задач сегментации может подойти SAM2, а для детекции - Grounding DINO.

	\begin{comment}
	\begin{figure}[H] % Позиционирование (здесь, верх, низ, страница)
	\centering % Центрирование
	\begin{tikzpicture}[every node/.style={align=center}]
		% Узлы схемы
		\node (start) [startstop] {Создание датасета};
		\node (train) [block, below=of start] {Обучение модели};
		\node (test) [decision, below=of train] {
			Модель\\
			работает удовлетворительно\\
			на всех фоноцелевых\\
			обстановках?
		};
		\node (fix) [block, right=of test] {
            Пополнение датасета\\
			недостяющими случаями
		};
		\node (end) [startstop, below=of test] {Конец};
		% Соединения
		\draw [arrow] (start) -- (train) coordinate[midway] (totrain);
        \draw [arrow] (train) -- (test);
		\draw [arrow] (test) -- node[anchor=south] {Нет} (fix);
		\draw [arrow] (test) -- node[anchor=east] {Да} (end);
		\draw [arrow] (fix) |- (totrain); % Г-образная стрелка
	\end{tikzpicture}
	\caption{Цикл улучшения данных с опорой на результат обучения}
	\label{fig:dataimprovementcycle}
	\end{figure}
	\end{comment}
\subsection{Формализация требований к отбору и разметке данных}
	Однако невозможно без достаточного опыта с первого раза составить исчерпывающий список требований, предъявляемый к данным. Это продиктовано итеративным подходом в самой разработке ML-проектов:
\subsection{Организация сбора первичных данных}
\subsection{Фильтрация данных по релевантности}
\subsection{Разметка данных}
\subsection{Конвертация данных в итоговый датасет}
\subsection{Подбор параметров аугментации}

\section{Обучение модели}
\subsection{Определение гиперпараметров модели в первом приближении с учётом аппаратных возможностей бортового вычислителя (рабочее разрешение изображения, базовая модель, сложность архитектуры и т.п.)}
\subsection{Сборка первой (ещё необученной) версии модели, её конвертация в ONNX и измерение fps её инференса на устройстве (коррекция гиперпараметров в случае необходимости)}
\subsection{Создание кода обучения модели}
\subsection{Запуск серии пристрелочных экспериментов (сужение размера пространства гиперпараметров)}
\subsection{Выбор метрик и оценка их объективности путём сравнения с результатами визуализации работы модели}
\subsection{Разработка и запуск кода обучения гипермодели для поиска субоптимального набора гиперпараметров (опционально)}
\subsection{До- или повторное обучение модели с лучшими гиперпараметрами}
\subsection{Фиксация версии модели с кратким описанием в таблице, доступной для команды плюсовиков (необходимо для версионирования конечного продукта и прозрачности дальнейших испытаний)}
\subsection{Конвертация в ONNX и передача модели в сопровождении описания алгоритма пред- и постобработки данных для корректной интеграции в прод}

\section{Тестирование}
\subsection{В симуляторе (симуляция / in silico)}
Преимущества симулятора:
\begin{enumerate}
	\item возможность испытать работу в замкнутом контуре не только модель, но и всю систему целиком;
	\item быстрый и дешёвый способ воспроизвести и отладить проблемы, выявленные на натурных и полунатурные испытаниях;
\end{enumerate}

\subsection{На заводе (полунатурные испытания / in vitro)}
\subsection{В полях (натурные испытания / in vivo)}