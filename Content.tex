\section{Цели и задачи проекта}
Главный проект декомпозируется представителями всех команд на составляющие, одной из которых и является ML-часть, выделяющаяся в отдельный Git-проект. ML-проект так же должен быть разбит на элементы. В идеале \textit{\textbf{составляющие ML-проекта должны быть сведены к давно известным задачам, решение которых является стереотипным}}: детекция, трекинг, сегментация, классификация, регрессия, фильтрация и т.п.

\subsection{Классификация ML-задачи}
От типа решаемой ML-задачи зависят дальнейшие шаги. Без этого невозможно даже формализовать требования к разметке, не говоря уже о последующих этапах.

В первом приближении декомпозиция ML-проекта должна сводить его элементы к использованию хорошо известных и успешно апробированных в мировой практике решений. Любое отклонение от этого переводит проект из разряда чисто инженерных в исследовательские, что влечёт дополнительные риски. Всё это необходимо учитывать ещё на этапе декомпозиции главной задачи.

Стереотипными CV-задачами являются: сегментация, детекция, трекинг, фильтрация, классификация, регрессия и т.п.

\subsection{Выбор модели подходящего семейства}
Если ML-задачу удалось свести к стереотипному решению, то оно автоматически диктует: архитектуру модели, пред- и постобработки данных, метрики и открытые датасеты. Например, для сегментации подходят U-Net или DeepLab, а для детекции - YOLO или R-CNN.Так же в большинстве случаев имеется возможность выбирать базовую модель (backbone) для извлечения признаков из изображений. Однако, использование моделей на базе трансформеров обычно не представляется возможным ввиду ограничений бортового оборудования и необходимости работы в реальном времени. Но определять модель настолько детально на этом этапе не требуется, т.к. впереди должны быть ещё исследования соотношений точности/скорости для различных гиперпараметров. Следует лишь иметь это ввиду и учитывать возможности выбранной аппаратной платформы.

Если для ML-задачи было выбрано нетипичное решение, следует выделить дополнительные ресурсы на проведение исследований. Это в любом случае потребует времени и резко увеличит риски провалить проект или завершить его позже планируемого срока.

\section{Создание датасета}
Большинство проблем почти любого ML-проекта (особенно на ранних стадиях) связано с недостатками датасета:
\begin{enumerate}
	\item беден даже с учётом аугментации или вообще отсутствует;
	\item нерелевантен (не в полной мере соответствует задаче, условиям наблюдения, типу используемых датчиков и т.п.);
	\item противоречиво, неполно или неточно размечен;
\end{enumerate}
Сам процесс сбора новых данных и их разметка так же могут вызывать трудности: 
\begin{enumerate}
	\item дороговизна получения новых данных;
	\item привязка к независящим от человека условиям (например, погода) или редкость фиксируемого явления;
	\item высокие требования к квалификации разметчика или трудоёмкость самой разметки;
	\item юридические ограничения (персональные данные, секретность и т.п.).
\end{enumerate}

Для компенсации указанных проблем зачастую прибегают к компьютерному моделированию. Это может быть разработка полноценного симулятора с нуля или модификация какого-то из существующих решений. К тому же \textit{\textbf{полноценный симулятор снимает необходимость размечать синтетические данные и открывает большое количество возможностей по тестированию}}. Но и это направление сопряжено с рядом проблем:
\begin{enumerate}
	\item необходимость иметь команду соответствующих разработчиков в штате или на аутсорсе;
	\item труднодостижимый фотореализм;
	\item сложная механика моделирования динамических систем, особенно относящихся к передвижению носителя или ОИ;
	\item некоторым задачам требуется большое количество 3D-моделей для наполнения фона или разнообразия ОИ, которые не всегда можно найти в открытых асетах.
\end{enumerate}
Если разработать полноценный симулятор нет возможности, то прибегают к созданию относительно релевантного набора данных, синтезированного иным образом. Например, с использованием какой-либо компьютерной игры или генеративной НС. В крайнем случае модифицируют имеющиеся данные в полуручном режиме. Но при всех этих вариантах, во-первых, недоступно полноценное моделирование работы замкнутой системы, а во-вторых, сохраняется необходимость размечать данные в ручном режиме.


\subsection{Формализация требований к отбору и разметке данных}
\subsection{Организация сбора первичных данных}
\subsection{Фильтрация данных по релевантности}
\subsection{Разметка данных}
\subsection{Конвертация данных в итоговый датасет}
\subsection{Подбор параметров аугментации}

\section{Обучение модели}
\subsection{Определение гиперпараметров модели в первом приближении с учётом аппаратных возможностей бортового вычислителя (рабочее разрешение изображения, базовая модель, сложность архитектуры и т.п.)}
\subsection{Сборка первой (ещё необученной) версии модели, её конвертация в ONNX и измерение fps её инференса на устройстве (коррекция гиперпараметров в случае необходимости)}
\subsection{Создание кода обучения модели}
\subsection{Запуск серии пристрелочных экспериментов (сужение размера пространства гиперпараметров)}
\subsection{Выбор метрик и оценка их объективности путём сравнения с результатами визуализации работы модели}
\subsection{Разработка и запуск кода обучения гипермодели для поиска субоптимального набора гиперпараметров (опционально)}
\subsection{До- или повторное обучение модели с лучшими гиперпараметрами}
\subsection{Фиксация версии модели с кратким описанием в таблице, доступной для команды плюсовиков (необходимо для версионирования конечного продукта и прозрачности дальнейших испытаний)}
\subsection{Конвертация в ONNX и передача модели в сопровождении описания алгоритма пред- и постобработки данных для корректной интеграции в прод}

\section{Тестирование}
\subsection{В симуляторе (симуляция / in silico)}
Преимущества симулятора:
\begin{enumerate}
	\item возможность испытать работу в замкнутом контуре не только модель, но и всю систему целиком;
	\item быстрый и дешёвый способ воспроизвести и отладить проблемы, выявленные на натурных и полунатурные испытаниях;
\end{enumerate}

\subsection{На заводе (полунатурные испытания / in vitro)}
\subsection{В полях (натурные испытания / in vivo)}